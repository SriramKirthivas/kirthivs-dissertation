\documentclass[12pt, a4paper]{article}

% Packages
\usepackage[utf8]{inputenc}
\usepackage[margin=1in]{geometry}
\usepackage{amsmath, amssymb}
\usepackage{graphicx}
\usepackage{hyperref}
\usepackage{xcolor}

% Title Information
\title{My Dissertation Learning Journal}
\author{Sriram Kirthivas (kirthivs@tcd.ie)}
\date{Started: October 14, 2025}

\begin{document}

\maketitle
\tableofcontents
\newpage

\section{October 2025 - Protocol Fundamentals}

\subsection{Week 1: October 14-21, 2025}

\textbf{Focus Area:} TLS Protocol Fundamentals

\textbf{What I learned:}
\begin{itemize}
    \item \textbf{TLS (Transport Layer Security):} A cryptographic protocol designed to provide security over computer networks such as the Internet
    \item TLS primarily provides privacy through use of cryptography such as certificates
    \item TLS runs in the presentation layer and is composed of TLS Record Protocol and TLS Handshake Protocol
    \item The handshake protocol establishes the cryptographic parameters for the session
    \item TLS 1.3 is the current version with improved security and performance over earlier versions
\end{itemize}

\textbf{Key Concepts:}
\begin{itemize}
    \item Symmetric vs. asymmetric encryption in TLS
    \item Certificate chain validation
    \item Perfect Forward Secrecy (PFS)
\end{itemize}

\textbf{Questions/To Review:}
\begin{itemize}
    \item How do certificate authorities (CAs) validate domain ownership?
    \item What are the differences between TLS 1.2 and TLS 1.3?
    \item How does TLS relate to SSH and SMTP in terms of security guarantees?
\end{itemize}

\vspace{1cm}
\hrule
\vspace{1cm}

\subsection{Week 2: October 22-28, 2025}

\textbf{Focus Area:} SSH Protocol and SMTP Basics

\textbf{What I learned:}
\begin{itemize}
    \item \textbf{SSH (Secure Shell):} A cryptographic network protocol for secure remote login and command execution
    \item SSH uses public-key cryptography for authentication
    \item SSH host keys are used to verify server identity (relevant for key re-use detection)
    \item \textbf{SMTP (Simple Mail Transfer Protocol):} The protocol used for email transmission between mail servers
    \item SMTP runs on port 25 by default (the port we'll be scanning)
    \item STARTTLS command upgrades plaintext SMTP connections to encrypted TLS
\end{itemize}

\textbf{Key Concepts:}
\begin{itemize}
    \item SSH key fingerprinting methods (MD5, SHA256)
    \item SMTP command sequence: HELO/EHLO, MAIL FROM, RCPT TO, DATA
    \item Opportunistic TLS vs. mandatory TLS in email
\end{itemize}

\textbf{Relevance to Dissertation:}
\begin{itemize}
    \item Port 25 scanning will identify SMTP servers in Irish IPv4 space
    \item TLS certificates from STARTTLS connections will be analyzed for key re-use
    \item Understanding SMTP is essential for responsible scanning practices
\end{itemize}

\textbf{Questions/To Review:}
\begin{itemize}
    \item What percentage of mail servers support STARTTLS?
    \item How are SMTP server certificates typically managed in practice?
    \item What are common misconfigurations that lead to key re-use?
\end{itemize}

\vspace{1cm}
\hrule
\vspace{1cm}

\subsection{Week 3: October 29 - November 4, 2025}

\textbf{Focus Area:} TLS/SMTP Integration and Certificate Basics

\textbf{What I learned:}
\begin{itemize}
    \item How SMTP servers negotiate TLS through STARTTLS command
    \item X.509 certificate structure and fields (subject, issuer, validity, public key)
    \item Self-signed certificates vs. CA-signed certificates in mail servers
    \item Many mail servers use self-signed certificates, making key re-use analysis important
    \item Certificate Subject Alternative Names (SANs) for multiple domains
\end{itemize}

\textbf{Key Concepts:}
\begin{itemize}
    \item Certificate fingerprinting using SHA-256 hash
    \item Public key extraction from certificates
    \item Distinction between certificate re-use and key re-use
\end{itemize}

\textbf{Practical Exercises:}
\begin{itemize}
    \item Used \texttt{openssl s\_client} to connect to mail servers and inspect certificates
    \item Practiced extracting public keys from certificates using OpenSSL commands
    \item Examined certificate chains from major email providers
\end{itemize}

\textbf{Questions/To Review:}
\begin{itemize}
    \item What is the average lifespan of mail server certificates in the wild?
    \item How common are wildcard certificates in email infrastructure?
    \item Review the "Clusters of Re-used Keys" paper methodology
\end{itemize}

\vspace{1cm}
\hrule
\vspace{1cm}

\section{November 2025 - Scanning Approaches and Research}

\subsection{Week 4: November 5-11, 2025}

\textbf{Focus Area:} Internet-Wide Scanning Concepts

\textbf{What I learned:}
\begin{itemize}
    \item \textbf{Internet-wide scanning:} Systematic probing of IPv4 address space to identify hosts and services
    \item Difference between horizontal scans (one port, many hosts) and vertical scans (many ports, one host)
    \item Our dissertation will use horizontal scanning on port 25 (SMTP)
    \item \textbf{Local scanning approach:} Focusing on Irish IPv4 space rather than global Internet
    \item Ethical considerations: rate limiting, respecting robots.txt, responsible disclosure
    \item Legal framework: Irish and EU regulations on network scanning
\end{itemize}

\textbf{Key Concepts:}
\begin{itemize}
    \item IPv4 address space: approximately 4.3 billion addresses
    \item Irish IPv4 allocation: much smaller subset, needs identification via RIR data
    \item RIPE NCC as the Regional Internet Registry for Europe
    \item Autonomous System Numbers (ASNs) for identifying Irish networks
\end{itemize}

\textbf{Research Papers Started:}
\begin{itemize}
    \item "Clusters of Re-used Keys" paper - initial reading
    \item ZMap paper - "Fast Internet-Wide Scanning and Its Security Applications"
    \item Notes on responsible scanning practices from security research community
\end{itemize}

\textbf{Next Steps:}
\begin{itemize}
    \item Deep dive into "Clusters of Re-used Keys" paper methodology
    \item Identify Irish IPv4 address ranges from RIPE database
    \item Research cryptographic key management best practices
    \item Begin outlining scanning methodology for dissertation
\end{itemize}

\textbf{Questions/To Review:}
\begin{itemize}
    \item How many IPv4 addresses are allocated to Ireland?
    \item What scan rate is considered ethical and non-disruptive?
    \item What are the legal requirements for notification before scanning?
    \item How do we handle complaints or takedown requests during scanning?
\end{itemize}

\vspace{1cm}
\hrule
\vspace{1cm}

\end{document}