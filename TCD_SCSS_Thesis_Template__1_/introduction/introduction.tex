\chapter{Introduction - Chapter}

\section{Section}

\subsection{Subsection}

\subsubsection{SSH}

SSH \cite{rfc4253} : \\
SSH -> transport layer -> secure, low-level transport protocol. \\
Provides strong encryption, crypto based host authentication and integrity protection. \\
Authentication -> host-based authentication. This does not perform user authentication. \\
Simple and flexible -> to allow parameter negotiation and to minimize number of round trips. 
Key exchange method, public key algo, symmetric encryption and others are negotiated. \\
Most env -> 2 round-trips needed for full key exchange, server authentication, service request and accept. 
Worst case - 3 round trips.

\paragraph{Connection setup :}

Works over any clean 8-bit binary-transparent transport. 
Transport should protect ssh connections against transmission errors. Client Initiates.

tcp/ip - listens to port 22(ssh). \\
After connection establishment - client and server must send an identification string.

\subparagraph{Identification string :}

SSH-protoversion-softwareversion SP comments CR LF

Protoversion - 2.0, comments - optional, if comments is used, SP should be used to separate softwareversion and comments. 
Identification - terminated by a single Carriage Return(CR) and single Line Feed(LF). 
NULL characters must not be sent. Max length - 255 characters including CR and LF.

Server - may send other lines of data before sending version string. 
Each line should be terminated by a CR and LF. 
Such lines should not begin with SSH- and should be encoded in ISO-10646 UTF-8. 
Client must be able to process such lines. 
If they are displayed, control character filtering should be used. 
Primary use is to allow TCP-Wrappers to display error message before disconnecting.

Protoversion and softwareversion - consist of printable US-ASCII characters, with exception of whitespace and minus sign.
Softwareversion - used to trigger compatibility extensions and to indicate capabilities of implementation. \\
Example : SSH-2.0-billsSSH\_3.6.3q3\textbackslash CR\textbackslash LF \\
Key exchange will begin immediately after sending this. 
All packets will use the binary packet protocol.

\subsubsection{TLS}

TLS \cite{rfc8446} \\
Transport Layer Security Protocol - primary goal - provide secure channel b/w 2 communicating peers. 
Only req - reliable, in-order data stream.

Properties of secure channel : \\
Authentication - server side always authenticated, client side optional. 
Authentication - via asymmetric crypto like RSA, Elliptic Curve Digital Signature Algo (ECDSA) or 
Edwards-Curve Digital Signature Algo (EdDSA) or symmetric pre-shared key (PSK). \\
Confidentiality - Data sent over the channel is only visible to endpoints. 
TLS does not hide the length of data it transmits, though endpoints may pad TLS records. \\
Integrity - Data sent cannot be modified by attackers without detection.

TLS has 2 primary components: \\
Handshake protocol - authenticates communicating parties, negotiates crypto modes and parameters, 
establishes shared keying material. Resists tampering by attackers. \\
Record protocol - uses parameters established by handshake to protect traffic. 
Divides traffic into records independently protected with keys.

TLS - application protocol independent. 
Does not specify how protocols add security with TLS, how to initiate handshaking or interpret certificates. 
Left to protocol designers running on top of TLS.

\subsubsection{SMTP}

SMTP \cite{rfc5321} \\
Objective - transfer mail reliably and efficiently. \\
SMTP - independent of transmission subsystem and requires only reliable ordered data stream channel. \\
Important feature - capability to transport mail across multiple networks, referred to as “SMTP mail relaying”.

Network examples: mutually-TCP-accessible hosts on public Internet, TCP/IP Intranet, LAN, WAN.

A process can transfer mail to another using relay or gateway processes. 
There can be multiple intermediate relay hosts.

SMTP client with a message establishes a two-way channel to an SMTP server. 
SMTP is responsible for transferring mail to servers or reporting failure.

How messages are presented and domain identifiers are determined is a local matter. 
Domains may be final or intermediate destinations. 
SMTP clients that forward all traffic blindly or do not maintain retry queues may conform to the spec 
but are not fully capable. 
Fully-capable SMTP implementations support queuing, retrying, and alternate address functions.

\section{Aims}
