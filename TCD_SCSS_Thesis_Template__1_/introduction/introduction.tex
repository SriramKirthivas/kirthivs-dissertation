\chapter{Introduction - Chapter}

\section{Section}

\subsection{Subsection}

\subsubsection{SSH}

SSH \cite{rfc4253} : \\
SSH -> transport layer -> secure, low-level transport protocol. \\
Provides strong encryption, crypto based host authentication and integrity protection. \\
Authentication -> host-based authentication. This does not perform user authentication. \\
Simple and flexible -> to allow parameter negotiation and to minimize number of round trips. 
Key exchange method, public key algo, symmetric encryption and others are negotiated. \\
Most env -> 2 round-trips needed for full key exchange, server authentication, service request and accept. 
Worst case - 3 round trips.

\paragraph{Connection setup :}

Works over any clean 8-bit binary-transparent transport. 
Transport should protect ssh connections against transmission errors. Client Initiates.

tcp/ip - listens to port 22(ssh). \\
After connection establishment - client and server must send an identification string.

\subparagraph{Identification string :}

SSH-protoversion-softwareversion SP comments CR LF

Protoversion - 2.0, comments - optional, if comments is used, SP should be used to separate softwareversion and comments. 
Identification - terminated by a single Carriage Return(CR) and single Line Feed(LF). 
NULL characters must not be sent. Max length - 255 characters including CR and LF.

Server - may send other lines of data before sending version string. 
Each line should be terminated by a CR and LF. 
Such lines should not begin with SSH- and should be encoded in ISO-10646 UTF-8. 
Client must be able to process such lines. 
If they are displayed, control character filtering should be used. 
Primary use is to allow TCP-Wrappers to display error message before disconnecting.

Protoversion and softwareversion - consist of printable US-ASCII characters, with exception of whitespace and minus sign.
Softwareversion - used to trigger compatibility extensions and to indicate capabilities of implementation. \\
Example : SSH-2.0-billsSSH\_3.6.3q3\textbackslash CR\textbackslash LF \\
Key exchange will begin immediately after sending this. 
All packets will use the binary packet protocol.

\subsubsection{TLS}

TLS \cite{rfc8446} \\
Transport Layer Security Protocol - primary goal - provide secure channel b/w 2 communicating peers. 
Only req - reliable, in-order data stream.

Properties of secure channel : \\
Authentication - server side always authenticated, client side optional. 
Authentication - via asymmetric crypto like RSA, Elliptic Curve Digital Signature Algo (ECDSA) or 
Edwards-Curve Digital Signature Algo (EdDSA) or symmetric pre-shared key (PSK). \\
Confidentiality - Data sent over the channel is only visible to endpoints. 
TLS does not hide the length of data it transmits, though endpoints may pad TLS records. \\
Integrity - Data sent cannot be modified by attackers without detection.

TLS has 2 primary components: \\
Handshake protocol - authenticates communicating parties, negotiates crypto modes and parameters, 
establishes shared keying material. Resists tampering by attackers. \\
Record protocol - uses parameters established by handshake to protect traffic. 
Divides traffic into records independently protected with keys.

TLS - application protocol independent. 
Does not specify how protocols add security with TLS, how to initiate handshaking or interpret certificates. 
Left to protocol designers running on top of TLS.

\subsubsection{SMTP}

SMTP \cite{rfc5321} \\
Objective - transfer mail reliably and efficiently. \\
SMTP - independent of transmission subsystem and requires only reliable ordered data stream channel. \\
Important feature - capability to transport mail across multiple networks, referred to as "SMTP mail relaying".

Network examples: mutually-TCP-accessible hosts on public Internet, TCP/IP Intranet, LAN, WAN.

A process can transfer mail to another using relay or gateway processes. 
There can be multiple intermediate relay hosts.

SMTP client with a message establishes a two-way channel to an SMTP server. 
SMTP is responsible for transferring mail to servers or reporting failure.

How messages are presented and domain identifiers are determined is a local matter. 
Domains may be final or intermediate destinations. 
SMTP clients that forward all traffic blindly or do not maintain retry queues may conform to the spec 
but are not fully capable. 
Fully-capable SMTP implementations support queuing, retrying, and alternate address functions.

\section{Cluster of Re-Used Keys} 

The research paper titled \textit{"Cluster of Re-Used Keys"}\cite{cryptoeprint:2018/299} surveys long-term cryptographic public keys used for TLS and SSH protocols across hosts in ten countries. The hosts examined were primarily those running SMTP, indicating an interest in measuring the security of email and related services. The primary finding is that key re-use is widespread across multiple IP addresses and even across different Autonomous Systems (ASes).

\subsection{Definition of a Cluster}
A \textbf{cluster} is defined as a set of IP addresses where each host shares at least one public key with another host in the same set.

\subsection{Key Re-Use Observations}

From a scan of 18{,}268 hosts in Ireland, approximately \textbf{53\%} of hosts running a cryptographic service were using public keys that were also observed on another IP address.  
Out of 54{,}447 host/port combinations running cryptographic protocols in the Irish scan, only \textbf{36\%} (20{,}053 entries) used unique keys—highlighting substantial key re-use.

The scan also identified a total of \textbf{1{,}437 clusters}, with the largest cluster containing \textbf{1{,}991 hosts}.

\subsection{Security Risks of Key Re-Use}

Key re-use introduces several undesirable security and privacy dependencies among cluster members:

\paragraph{Masquerade}
A breach on any one host allows an attacker to impersonate any other host within the same cluster.

\paragraph{Increased Leak Risk}
The probability of a private key leak increases significantly; compromise of one host exposes all other hosts sharing the same key.

\paragraph{Credential Capture}
An attacker masquerading as a legitimate service can capture sensitive credentials such as IMAP or SMTP passwords.

\paragraph{Web Origin Policy Breach}
If clustered hosts belong to different web origins, a compromise allows attackers to steal HTTP cookies, bypassing the browser's origin policy.

\subsection{Causes of Key Sharing}

Common reasons behind widespread key re-use include:

\begin{itemize}
    \item A single host being assigned multiple IP addresses
    \item Redundant mirrored hosts
    \item Cloned virtual machines containing pre-installed host keys
    \item Large-scale use of wildcard certificates
    \item Vendors shipping products with default key pairs
    \item Operational laziness or misconfiguration by service operators
\end{itemize}

\section{ZMap Network Scanning Tool}

ZMap is a fast, stateless, single-packet network scanner designed for Internet-wide network surveys \cite{zmap2013}. On typical hardware, ZMap is capable of scanning the entire public IPv4 address space on a single port in under 45 minutes. For example, it can send a TCP SYN packet to every IPv4 address on port~25 to identify potential SMTP servers.

ZMap supports GNU/Linux, macOS, and BSD, and currently implements probe modules for TCP SYN scans, ICMP probes, and DNS queries.

\subsection{Performance and Limitations}

ZMap transmits packets as quickly as the host network interface allows. Since it intentionally does \textbf{not} implement congestion control, two main risks arise:

\paragraph{Target Network DoS}
Scanning a small subnet at an excessively high rate may unintentionally cause a denial-of-service (DoS). It is recommended not to run ZMap at speeds of 1~Gbps or higher when scanning small networks; instead, keep rates below 10~Mbps to avoid overwhelming the target.

\paragraph{Source Network Overload}
ZMap may also overload the source network. Some switches and routers cannot handle high-speed traffic consisting of many small packets.

ZMap scans in-scope IP addresses in a random order to reduce localized impact on target networks. The load experienced by a subnet depends on both the configured sending rate and the size of the target IP range.

Potential consequences of overloading a network include:
\begin{itemize}
    \item network administrators blocking the scanning IP address,
    \item routers silently dropping scan traffic,
    \item disruption of local users due to bandwidth starvation.
\end{itemize}

\subsection{Execution Model}

By default, ZMap uses four threads, unless the host machine has fewer than four CPU cores. The recommended number of threads for ZMap is:
\[
T_{\text{required}} = T + 2,
\]
where $T$ is the number of sending threads.

\subsection{Basic Usage}

To perform an initial scan that sends TCP SYN packets to all IP addresses in a subnet \texttt{xx.xx.xx.xx/xx} on port~80 at a send rate of 128~packets/s, use:

\begin{verbatim}
sudo zmap -p 80 -r 128 xx.xx.xx.xx/xx
\end{verbatim}

ZMap produces two types of output:
\begin{enumerate}
    \item a list of IP addresses that responded, and
    \item periodic scan status messages printed every second.
\end{enumerate}

To save scan results to a CSV file, use:

\begin{verbatim}
sudo zmap -p 80 -o output.csv -r 128 xx.xx.xx.xx/xx
\end{verbatim}

\subsection{ZMap Command Parameters}

\begin{center}
\begin{tabular}{|l|l|}
\hline
\textbf{Parameter} & \textbf{Description} \\
\hline
\texttt{-p} & Target port number \\
\hline
\texttt{-n} & Number of hosts to scan \\
\hline
\texttt{-r} & Send rate (packets per second) \\
\hline
\texttt{-o} & Output file for results \\
\hline
\texttt{-w} & Whitelist file (IP ranges to scan) \\
\hline
\texttt{-b} & Blacklist file (IP ranges to exclude) \\
\hline
\end{tabular}
\end{center}

\section{VPS Setup with Hostinger}

For internet-scale scanning, a VPS with outbound port 25 access is required. Most networks block port 25 to prevent spam. Hostinger VPS has outbound port 25 enabled by default.

\subsection{Selecting VPS Plan}

\begin{enumerate}
    \item Go to \url{https://www.hostinger.com/vps-hosting}
    \item Select KVM 2 or higher plan
    \item Choose Ubuntu 22.04 LTS
    \item Complete purchase and wait for provisioning
\end{enumerate}

\subsection{Initial Setup}

Connect to VPS via SSH:

\begin{verbatim}
ssh root@<vps_ip_address>
\end{verbatim}

Update system:

\begin{verbatim}
apt update && apt upgrade -y
\end{verbatim}

\subsection{Verify Port 25 Connectivity}

Hostinger has outbound port 25 enabled by default. Test with:

\begin{verbatim}
nc -vz smtp.gmail.com 25
\end{verbatim}

Expected output:

\begin{verbatim}
Connection to smtp.gmail.com 25 port [tcp/smtp] succeeded!
\end{verbatim}

\subsection{Install ZMap}

\begin{verbatim}
apt install -y zmap
\end{verbatim}

Verify installation:

\begin{verbatim}
zmap --version
\end{verbatim}

\subsection{Clone Surveys Repository}

\begin{verbatim}
cd ~
git clone https://github.com/sftcd/surveys.git
cd ~/surveys
\end{verbatim}

Repository contains scripts for scanning and key reuse analysis: IPsFromMM.py, SameKeys.py, FreshGrab.py, etc.

\section{Extracting Irish IP Ranges from MaxMind}

The surveys repository includes scripts for downloading MaxMind databases and extracting country-specific IP ranges.

\subsection{MaxMind Account Setup}

\begin{enumerate}
    \item Go to \url{https://www.maxmind.com/en/geolite2/signup}
    \item Create free account
    \item Go to Account -> Manage License Keys
    \item Generate and save license key
\end{enumerate}

\subsection{Update MaxMind Database}

The repository includes \texttt{mm\_update.sh} script to download MaxMind GeoLite2 databases. The databases are stored in the \texttt{mmdb/} directory within the repository.

\begin{verbatim}
cd ~/surveys
export MAXMIND_LICENSE_KEY="YOUR_LICENSE_KEY"
./mm_update.sh
\end{verbatim}

This downloads:
\begin{itemize}
    \item GeoLite2-ASN database
    \item GeoLite2-City database
    \item GeoLite2-Country database
\end{itemize}

\subsection{Extract Ireland IP Ranges Using IPsFromMM.py}

The \texttt{IPsFromMM.py} script extracts IP prefixes for a specified country from the MaxMind database. It takes the two-letter country code as input (e.g., "IE" for Ireland).

\begin{verbatim}
cd ~/surveys
python3 IPsFromMM.py -c IE
\end{verbatim}

The script:
\begin{enumerate}
    \item Reads the country code (IE for Ireland)
    \item Queries the MaxMind database for matching prefixes
    \item Outputs IPv4 prefixes to \texttt{mm-ips.IE.v4}
    \item Outputs IPv6 prefixes to \texttt{mm-ips.IE.v6}
\end{enumerate}

\subsection{Verify Output}

Check the generated IPv4 prefix file:

\begin{verbatim}
head mm-ips.IE.v4
\end{verbatim}

Sample output:
\begin{verbatim}
2.18.232.0/21
2.21.173.0/24
2.58.8.0/22
5.62.43.224/27
5.62.61.76/30
5.83.224.0/20
5.133.176.0/21
5.149.224.0/21
5.152.192.0/19
5.178.64.0/19
\end{verbatim}

Count total blocks:
\begin{verbatim}
wc -l mm-ips.IE.v4
\end{verbatim}

\subsection{Summary}

Setup completed:
\begin{itemize}
    \item Hostinger VPS with Ubuntu 22.04
    \item Port 25 enabled by default, verified with \texttt{nc -vz smtp.gmail.com 25}
    \item ZMap installed
    \item Surveys repository cloned from \url{https://github.com/sftcd/surveys}
    \item MaxMind database updated using \texttt{mm\_update.sh}
    \item Ireland IP list extracted using \texttt{IPsFromMM.py} and stored in \texttt{mm-ips.IE.v4}
\end{itemize}

Ready to scan Irish IPs:
\begin{verbatim}
sudo zmap -p 25 -w mm-ips.IE.v4 -o ireland_p25.csv -r 1000
\end{verbatim}
